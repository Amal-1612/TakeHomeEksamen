\section*{Opgave 1}

\\
\\
I det følgende lader vi U = \{1, 2, 3, ..., 15\} være universet (universal set).
\\
Betragt de to mængder
\begin{center}
    \( A = \{ 2n \mid n \in S \} \) og \( B = \{ 2n + 2 \mid n \in S \} \)
\end{center}
\\
hvor \(S = \{1, 2, 3, 4\}.\)
\\
Angiv Samtlige elementer i hver af følgende mængder
\\
\\
\textbf{a) A}\\
Siden mængden A er defineret som \(\{ 2n \mid n \in S \}\), ved vi at vi kan indsætte elementerne fra mængden S på n's plads og tilføje dem til A, hvis resultatet er \(1 <= x <= 15\), hvilket vi ved fra vores univers.
\begin{center}
\(A = \{2*1, 2*2, 2*3, 2*4\} = \textbf{\{2, 4, 6, 8\}}\) \\
\end{center}
\\
\textbf{b) B}\\
Mængden B er defineret som \(B = \{2n + 2 \mid n \in S\}\), så vi kan igen indsætte de relevante elementer på de rigtige pladser. 
\begin{center}
    \(B = \{(3*1+2), (3*2+2), (3*3+2), (3*4+2)\} = \textbf{\{5, 8, 11, 14\}}\)
\end{center}
\textbf{c) A \(\cap\) B}\\
A intersection B refererer til mængden at de elementer som både er i A og B. Vi kan ud fra vores svar i del-opgaverne a) og b) se at det eneste der optræder i begge mængder er 8, hvilket giver os: 
\begin{center}
    \(A \cap B = \textbf{\{8\}}\) \\
\end{center}
\textbf{d) A \(\cup\) B}\\
A union B refererer til mængden af alle elementer i A og B, hvilket giver os: 
\begin{center}
    \(A \cup B = \textbf{\{2, 4, 5, 6, 8, 11, 14\}}\)
\end{center}
\textbf{e) A \( - \) B}\\
Mængdedifferensen mellem A og B refererer til mængden af elementer som optræder i A, men ikke B. Dette giver os mængden:
\begin{center}
    \(A - B = \textbf{\{2, 4, 6\}}\)
\end{center}
\textbf{f) \(\overline{A}\)} \\
Komplementet af A refererer til elementerne der optræder i universet, men \textit{ikke} er i A. Dette giver:
\begin{center}
    \(\overline{A} = \textbf{\{1, 3, 5, 7, 9, 10, 11, 12, 13, 14, 15\}}\)
\end{center}
\\
\\